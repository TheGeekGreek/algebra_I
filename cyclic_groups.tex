\subsection{Cyclic Groups}

\begin{definition}
	A group or subgroup is \emph{cyclic} when it is generated by a single element.
\end{definition}

\vspace{2mm}

\begin{lemma}
	Every group of prime order is cyclic.
\end{lemma}

\vspace{2mm}

\begin{proposition}
	Every subgroup of $(\mathbb{Z},+)$ is cyclic, generated by a unique nonnegative integer.
\end{proposition}

\begin{proof}
	Let $H \leq \mathbb{Z}$. If $H = \cbr[0]{0}$, then $H = \innerprod{0}$. So assume $H \neq \cbr[0]{0}$. Thus $H$ contains a nonzero integer. Since $H \leq \mathbb{Z}$ we have that $H$ contains a positive integer. Let us denote the smallest positive integer in $H$ by $n$. Every integer multiple of $n$ belongs to $H$. Conversly, if $m \in H$ we have by integer division $m = nq + r$ for some $q,r \in \mathbb{Z}$ with $0 \leq r < n$. So $m - nq = r \in H$ contradicting the minimality of $n$ being the smallest positive integer. Hence $m = nq$ and so $m \in \innerprod{n}$.
\end{proof}

\vspace{2mm}

\begin{proposition}
	Let $G$ be a group and let $a \in G$. If $a^m \neq 1$ for all $m \neq 0$, then $\langle a \rangle \cong \mathbb{Z}$; in particular $\langle a \rangle$ is infinite. Otherwise, there is a smallest positive integer $n$ such that $a^n = 1$, and then $a^m = 1$ if and only if $n$ divides $m$, and $\langle a \rangle \cong \mathbb{Z}/n\mathbb{Z}$; in particular, $\langle a \rangle$ is finite of order $n$.
\end{proposition}

\vspace{2mm}

\begin{corollary}
	Any two cyclic groups of order $n$ are isomorphic.
\end{corollary}

\vspace{2mm}

\begin{corollary}
	Every subgroup of a cyclic group is cyclic. Furthermore, either $H = \cbr[0]{1}$ or $H = \innerprod{x^n}$, where $n$ is the least positive integer with $x^n \in H$.
\end{corollary}

\begin{proof}
	Let $H \leq G := \innerprod{x}$, $H \neq \cbr[0]{1}$. Then $H' := \cbr[0]{k \in \mathbb{Z} : x^k \in H} \leq (\mathbb{Z},+)$ since 

	\begin{itemize}
		\item $0 \in H'$ by $1 = x^0 \in H \leq G$;
		\item If $k,k' \in H'$ then $k + k' \in H'$ by $x^{k + k'} = x^kx^{k'} \in H$;
		\item If $k \in H'$ then also $-k \in H'$ by $x^{-k} = \del[0]{x^{k}}^{-1} \in H$. 
	\end{itemize}

	Therefore $H' = \innerprod{n}$ for the least positive integer in $H'$. Now consider $\innerprod{x^n}$. We have that $\innerprod{x^n} \subseteq H$ by the previous observation and if $x^k \in H$ for some $k \in \mathbb{Z}$ we have $k \in H' = \innerprod{n}$ and so $k = mn$ for some $m \in \mathbb{Z}$ which yields $x^k = x^{mn} = \del[0]{x^n}^m \in \innerprod{x^n}$.    
\end{proof}

\vspace{2mm}

\begin{proposition}
	A cyclic group $G := \innerprod{x}$ of finite order $n$ has a unique subgroup of order $d$, namely $\langle x^{n/d} \rangle = \cbr[0]{g \in G : g^d = 1}$, for every divisor $d$ of $n$.	
\end{proposition}

\begin{proof}
	We prove the equality $\langle x^{n/d} \rangle = \cbr[0]{g \in G : g^d = 1}$ only. An element of $\langle x^{n/d} \rangle$ has the form $x^{kn/d}$ for some integer $k$. Therefore 

	\begin{equation*}
		\del[0]{x^{kn/d}}^d = (x^n)^k = 1^k = 1
	\end{equation*}

	Conversly if $g \in G$ with $g^d = 1$ we have that $g = x^k$ for some integer $k$ since $G$ is cyclic. Hence $1 = g^d = x^{kd}$ and so $n \mid kd$. So 

	\begin{equation*}
		g = x^k = (x^{kd/n})^{n/d} = (x^{n/d})^{kd/n} \in \langle x^{n/d}\rangle
	\end{equation*}
\end{proof}

\vspace{2mm}

\begin{lemma}
	For all $n \in \mathbb{N}$

	\begin{equation*}
		\mathbb{Z}^\times_n = \cbr[0]{\overline{k} :\gcd(n,k)=1} \qquad \text{and} \qquad \abs[0]{\mathbb{Z}^\times_n} = \varphi(n)
	\end{equation*}
\end{lemma}

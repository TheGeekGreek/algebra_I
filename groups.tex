\section{Groups}

\subsection{Group Actions}
What follows is based on \cite[99]{rose2009course}. 

\begin{itemize}
	\item Let $H \leq G$, define $X := \cbr[0]{xH : x \in G}$ and consider the transitive group action
		\begin{equation}
			\begin{cases}
				G \times X \to X\\
				(g,xH) \mapsto gxH
			\end{cases}
		\end{equation}

		The stabilizer of $xH \in X$ is given by 
		\begin{equation}
			G_{xH} = xHx^{-1}
		\end{equation}

		\noindent since if $g \in xHx^{-1}$ we have $g = xhx^{-1}$ for some $h \in H$ and thus $gxH = xhx^{-1}xH = xhH = xH$ and thus $g \in G_{xH}$. Conversly, if $g \in G_{xH}$ we have $gxH = xH$ and thus $gxh = xh'$ for some $h,h' \in H$ which implies $g = xh'h^{-1}x^{-1} \in xHx^{-1}$. Thus 
		\begin{equation}
			\ker \lambda = \bigcap_{xH \in X} G_{xH} = \bigcap_{x \in G}xHx^{-1}.
		\end{equation}

		If $[G:H]$ is finite, then $S_X \cong S_{\abs[0]{X}}$ and therefore by the isomorphism theorem and Lagrange
		\begin{equation}
			[G:\cap_{x \in G}xHx^{-1}] = \abs[0]{G/\cap_{x \in G}xHx^{-1}} \big| n!
		\end{equation}

		Assume $G$ is simple and $H < G$ with finite index. Then $\bigcap_{x \in G}xHx^{-1} = \langle 1 \rangle$ since if $\bigcap_{x \in G}xHx^{-1} = G$ we have $g \in xHx^{-1}$ for every $x \in G$ which implies $g \in H$ and thus $G = H$ which contradicts $H < G$. 
\end{itemize}


\subsection{Symmetric Groups}

\begin{itemize}
	\item The number $n_k$ of $k$-cycles in $S_n$ is given by 
		\begin{equation}
			n_k = \frac{n!}{k(n - k)!}.
		\end{equation}

	\item $A_n$ is generated by all $3$-cycles.
	\item For $n \geq 5$, $A_n$ is simple.
\end{itemize}

\subsection{Sylow Theorems}

Suppose $G$ is a finite group and $\abs[0]{G} = p^r m$, where $p \nmid m$. The number $n_p$ of $p$-Sylowgroups fulfills 
\begin{equation}
	n_p \mid m \qquad \text{and} \qquad n_p \in \cbr[0]{1 + kp : k \in \mathbb{N}_0}.
\end{equation}

The Sylow theorems are often used to show that a groups of a certain order cannot be simple, i.e. have no nontrivial normal subgroups. This is done by showing that there exists a unique $p$-Sylowgroup. Since if $H \leq G$ is of unique order, we have that $\iota_g(H) = H$ for any $g \in G$. Proving in general that a group is not simple may be difficult. But most of the time we end up having the oportunity $n_p \in \cbr[0]{1,n}$ where $n \in \mathbb{N}$ where $\abs[0]{G} = p^rm$ and $p \nmid m$. Often the following procedure works. Assume $n_p = n$ and let $X$ be the set of $p$-Sylowgroups. Consider the group action
	\begin{equation}
		\begin{cases}
			G \times X \to X\\
			(g,P) \mapsto gPg^{-1}
		\end{cases}.
	\end{equation}

	This action is well defined, since generally if $P$ is a $p$-Sylow group so is $gPg^{-1}$ for any $g \in G$ since $gPg^{-1} = \iota_g(P)$ where 
	\begin{equation}
		\iota_g:\begin{cases}
			G \to G\\
			x \mapsto gxg^{-1}
		\end{cases}
	\end{equation}
	
	\noindent is the so-called inner automorphism. Now if $\abs[0]{P} = p^r$ then $\abs[0]{gPg^{-1}} = p^r$ and clearly $gPg^{-1} \leq G$. Since $\abs[0]{X} = n$ we have $S_X \cong S_n$ as one trivially sees by considering the isomorphism 
	\begin{equation}
		\iota:\begin{cases} S_n \to S_X\\
			\sigma \mapsto \begin{pmatrix}x_1 & \dots & x_n\\
				x_{\sigma(1)} & \dots & x_{\sigma(n)}
			\end{pmatrix}
		\end{cases}
	\end{equation}

	Therefore by considering the permutation representation of the group action above 
	\begin{equation}
		\lambda:\begin{cases}
			G \to S_X\\
			g \mapsto \lambda_g
		\end{cases}
		\qquad
		\text{where} 
		\qquad 
		\lambda_g:\begin{cases}
			X \to X\\
			P \mapsto gPg^{-1}
		\end{cases}
	\end{equation}

	\noindent which is a homomorphism and using that the composition of homomorphisms is again a homomorphism, we get a homomorphism 
	\begin{equation}
		\lambda': G \to S_n
	\end{equation}

\subsection{Direct Products}

\begin{definition}
	Suppose $H,J \unlhd G$ with $H \cap J = \langle 1 \rangle$ and $G = HJ$. Then $G$ is said to be the \bld{internal direct product} of $H$ and $J$ and we have 
	\begin{equation}
		G \cong H \times J \cong J \times H.
	\end{equation}
\end{definition}

\begin{definition}
	Let $A \leq G$, $K \unlhd G$ where $G = AK$ and $A \cap K = \langle 1 \rangle$. Then $G$ is said to be an \bld{internal semi-direct product} of $K$ by $A$, written 
	\begin{equation}
		G \cong A \rtimes K.
	\end{equation}
\end{definition}

%%%%%%%%%%%%%%%%%%%%%%%%%%%%%%%%%%%%%%%%%%%%%%%%%%%%%%%%%%%%%%%%%%%%%%%%%%
%Author:																 %
%-------																 %
%Yannis Baehni at University of Zurich									 %
%baehni.yannis@uzh.ch													 %
%																		 %
%Version log:															 %
%------------															 %
%06/02/16 . Basic structure												 %
%04/08/16 . Layout changes including section, contents, abstract.		 %
%%%%%%%%%%%%%%%%%%%%%%%%%%%%%%%%%%%%%%%%%%%%%%%%%%%%%%%%%%%%%%%%%%%%%%%%%%

%Page Setup
\documentclass[
	11pt, 
	oneside, 
	a4paper,
	reqno,
	final
]{amsart}

\usepackage[
	left = 3cm, 
	right = 3cm, 
	top = 3cm, 
	bottom = 3cm
]{geometry}

%Headers and footers
\usepackage{fancyhdr}
	\pagestyle{fancy}
	%Clear fields
	\fancyhf{}
	%Header right
	\fancyhead[R]{
		\footnotesize
		Yannis B\"{a}hni\\
		\href{mailto:yannis.baehni@uzh.ch}{yannis.baehni@uzh.ch}
	}
	%Header left
	\fancyhead[L]{
		\footnotesize
		MAT211: Algebra I\\
		HS16
	}
	%Page numbering in footer
	\fancyfoot[C]{\thepage}
	%Separation line header and footer
	\renewcommand{\headrulewidth}{0.4pt}
	%\renewcommand{\footrulewidth}{0.4pt}
	
	\setlength{\headheight}{19pt} 

%Title
\usepackage[foot]{amsaddr}
%\usepackage{mathptmx}
\usepackage{xspace}
\makeatletter
\def\@textbottom{\vskip \z@ \@plus 1pt}
\let\@texttop\relax
\usepackage{etoolbox}
\patchcmd{\abstract}{\scshape\abstractname}{\textbf{\abstractname}}{}{}

%Switching commands for different section formats
%Mainsectionsytle
\newcommand{\mainsectionstyle}{%
  	\renewcommand{\@secnumfont}{\bfseries}
  	\renewcommand\section{\@startsection{section}{1}%
    	\z@{.5\linespacing\@plus.7\linespacing}{-.5em}%
    	{\normalfont\bfseries}}%
	\renewcommand\subsection{\@startsection{subsection}{2}%
    	\z@{.5\linespacing\@plus.7\linespacing}{-.5em}%
    	{\normalfont\bfseries}}%
	\renewcommand\subsubsection{\@startsection{subsubsection}{3}%
    	\z@{.5\linespacing\@plus.7\linespacing}{-.5em}%
    	{\normalfont\bfseries}}%
}
\newcommand{\originalsectionstyle}{%
\def\@secnumfont{\bfseries}%\mdseries
\def\section{\@startsection{section}{1}%
  \z@{.7\linespacing\@plus\linespacing}{.5\linespacing}%
  {\normalfont\bfseries\centering}}
	\renewcommand\subsection{\@startsection{subsection}{2}%
    	\z@{.5\linespacing\@plus.7\linespacing}{-.5em}%
    	{\normalfont\bfseries}}%
	\renewcommand\subsubsection{\@startsection{subsubsection}{3}%
    	\z@{.5\linespacing\@plus.7\linespacing}{-.5em}%
    	{\normalfont\bfseries}}%
}
%Formatting title of TOC
\renewcommand{\contentsnamefont}{\bfseries}
%Table of Contents
\setcounter{tocdepth}{3}

% Add bold to \section titles in ToC and remove . after numbers
\renewcommand{\tocsection}[3]{%
  \indentlabel{\@ifnotempty{#2}{\bfseries\ignorespaces#1 #2\quad}}\bfseries#3}
% Remove . after numbers in \subsection
\renewcommand{\tocsubsection}[3]{%
  \indentlabel{\@ifnotempty{#2}{\ignorespaces#1 #2\quad}}#3}
\let\tocsubsubsection\tocsubsection% Update for \subsubsection
%...

\newcommand\@dotsep{4.5}
\def\@tocline#1#2#3#4#5#6#7{\relax
  \ifnum #1>\c@tocdepth % then omit
  \else
    \par \addpenalty\@secpenalty\addvspace{#2}%
    \begingroup \hyphenpenalty\@M
    \@ifempty{#4}{%
      \@tempdima\csname r@tocindent\number#1\endcsname\relax
    }{%
      \@tempdima#4\relax
    }%
    \parindent\z@ \leftskip#3\relax \advance\leftskip\@tempdima\relax
    \rightskip\@pnumwidth plus1em \parfillskip-\@pnumwidth
    #5\leavevmode\hskip-\@tempdima{#6}\nobreak
    \leaders\hbox{$\m@th\mkern \@dotsep mu\hbox{.}\mkern \@dotsep mu$}\hfill
    \nobreak
    \hbox to\@pnumwidth{\@tocpagenum{\ifnum#1=1\bfseries\fi#7}}\par% <-- \bfseries for \section page
    \nobreak
    \endgroup
  \fi}
\AtBeginDocument{%
\expandafter\renewcommand\csname r@tocindent0\endcsname{0pt}
}
\def\l@subsection{\@tocline{2}{0pt}{2.5pc}{5pc}{}}
\def\l@subsubsection{\@tocline{2}{0pt}{4.5pc}{5pc}{}}
\makeatother

\advance\footskip0.4cm
\textheight=54pc    %a4paper
\textheight=50.5pc %letterpaper
\advance\textheight-0.4cm
\calclayout

%Font settings
%\usepackage{anyfontsize}
%Footnote settings
%\usepackage{mathptmx}
\usepackage{footmisc}
%	\renewcommand*{\thefootnote}{\fnsymbol{footnote}}
\usepackage{commath}
%Further math environments
%Further math fonts (loads amsfonts implicitely)
\usepackage{amssymb}
%Redefinition of \text
%\usepackage{amstext}
\usepackage{upref}
%Graphics
%\usepackage{graphicx}
%\usepackage{caption}
%\usepackage{subcaption}
%Frames
\usepackage{mdframed}
\allowdisplaybreaks
%\usepackage{interval}
\newcommand{\toup}{%
  \mathrel{\nonscript\mkern-1.2mu\mkern1.2mu{\uparrow}}%
}
\newcommand{\todown}{%
  \mathrel{\nonscript\mkern-1.2mu\mkern1.2mu{\downarrow}}%
}
\AtBeginDocument{\renewcommand*\d{\mathop{}\!\mathrm{d}}}
\renewcommand{\Re}{\operatorname{Re}}
\renewcommand{\Im}{\operatorname{Im}}
\DeclareMathOperator\Log{Log}
\DeclareMathOperator\Arg{Arg}
\DeclareMathOperator\sech{sech}
\DeclareMathOperator\Inn{Inn}
\DeclareMathOperator\Aut{Aut}
\DeclareMathOperator\Hom{Hom}
\DeclareMathOperator\id{id}
\newcommand{\innerprod}[1]{\left\langle {#1} \right\rangle}
%\usepackage{hhline}
%\usepackage{booktabs} 
%\usepackage{array}
%\usepackage{xfrac} 
%\everymath{\displaystyle}
%Enumerate
\usepackage{tikz}
\usetikzlibrary{external}
\tikzexternalize % activate!
\usetikzlibrary{patterns}
\pgfdeclarepatternformonly{adjusted lines}{\pgfqpoint{-1pt}{-1pt}}{\pgfqpoint{40pt}{40pt}}{\pgfqpoint{39pt}{39pt}}%
{
  \pgfsetlinewidth{.8pt}
  \pgfpathmoveto{\pgfqpoint{0pt}{0pt}}
  \pgfpathlineto{\pgfqpoint{39.1pt}{39.1pt}}
  \pgfusepath{stroke}
}
\usepackage{enumitem}

\definecolor{anti-flashwhite}{rgb}{0.95, 0.95, 0.96}
%\renewcommand{\labelitemi}{$\bullet$}
%\renewcommand{\labelitemii}{$\ast$}
%\renewcommand{\labelitemiii}{$\cdot$}
%\renewcommand{\labelitemiv}{$\circ$}
%Colors
%\usepackage{color}
%\usepackage[cmtip, all]{xy}
%Theorems
\newtheoremstyle{bold}              	 %Name
  {}                              %Space above
  {}                              %Space below
  {\itshape}		                     %Body font
  {}                                     %Indent amount
  {\scshape}                             %Theorem head font
  {.}                                    %Punctuation after theorem head
  { }                                    %Space after theorem head, ' ', 
  										 %	or \newline
  {} 
\theoremstyle{bold}
\newmdtheoremenv[%
  backgroundcolor=anti-flashwhite,
  bottomline=false,
  topline=false,
  rightline=false,
  leftline=false]{definition}{Definition}[section]
 \newmdtheoremenv[%
  backgroundcolor=anti-flashwhite,
  bottomline=false,
  topline=false,
  rightline=false,
  leftline=false]{proposition}{Proposition}[section]
\newmdtheoremenv[%
  backgroundcolor=anti-flashwhite,
  bottomline=false,
  topline=false,
  rightline=false,
  leftline=false]{lemma}{Lemma}[section]
\newmdtheoremenv[%
  backgroundcolor=anti-flashwhite,
  bottomline=false,
  topline=false,
  rightline=false,
  leftline=false]{theorem}{Theorem}[section]
\newmdtheoremenv[%
  backgroundcolor=anti-flashwhite,
  bottomline=false,
  topline=false,
  rightline=false,
  leftline=false]{corollary}{Corollary}[section]
\newtheorem*{definition*}{Definition}
%\newtheorem{definition}{Definition}[section]
\newtheorem*{lemma*}{Lemma}
%\newtheorem{lemma}{Lemma}[section]
\newtheorem{Proof}{Proof}[section]
%\newtheorem{proposition}{Proposition}[section]
\newtheorem{properties}{Properties}[section]
%\newtheorem{corollary}{Corollary}[section]
\newtheorem*{theorem*}{Theorem}
%\newtheorem{theorem}{Theorem}[section]
\newtheorem{example}{Example}[section]
\newtheorem*{remark*}{Remark}
\newtheorem{remark}{Remark}[section]
%German non-ASCII-Characters
%Graphics-Tool
%\usepackage{tikz}
%\usepackage{tikzscale}
%\usepackage{bbm}
%\usepackage{bera}
%Listing-Setup
%Bibliographie
\usepackage[backend=bibtex, style=alphabetic]{biblatex}
%\usepackage[babel, german = swiss]{csquotes}
\bibliography{Bibliography}
%PDF-Linking
%\usepackage[hyphens]{url}
\usepackage[bookmarksopen=true,bookmarksnumbered=true]{hyperref}
%\PassOptionsToPackage{hyphens}{url}\usepackage{hyperref}
\hypersetup{
  colorlinks   = true, %Colours links instead of ugly boxes
  urlcolor     = blue, %Colour for external hyperlinks
  linkcolor    = blue, %Colour of internal links
  citecolor    = blue %Colour of citations
}
%Weierstrass-P symbol for power set
\newcommand{\powerset}{\raisebox{.15\baselineskip}{\Large\ensuremath{\wp}}}

\newcommand{\hl}[1]{\textbf{#1}}
\begin{document}
\title{Alegbra I - Summary}
\author{Yannis B\"{a}hni}
\address[Yannis B\"{a}hni]{University of Zurich, R\"{a}mistrasse 71, 8006 Zurich}
\email[Yannis B\"{a}hni]{\href{mailto:yannis.baehni@uzh.ch}{yannis.baehni@uzh.ch}}

\maketitle
\thispagestyle{fancy}

\tableofcontents

\originalsectionstyle

\section{Groups}
%Subgroups
\subsection{Subgroups}
\begin{definition}
	A \emph{subgroup} of a group $G$ is a subset $H \subseteq G$ such that

	\begin{enumerate}[label = (\arabic*)]
		\item $1 \in H$\\
		\item $x \in H$ implies $x^{-1} \in H$\\
		\item $x,y \in H$ implies $xy \in H$
	\end{enumerate}
\end{definition}

\vspace{2mm}

\begin{proposition}
	$H \leq G$ if and only if $H \neq \varnothing$ and $x,y \in H$ implies $xy^{-1} \in H$.
\end{proposition}

\vspace{2mm}

\begin{proposition}
	For $H \neq \varnothing$ the following conditions are equivalent:
	
	\begin{enumerate}[label = (\arabic*)]
		\item $H \leq G$\\
		\item $HH \subseteq H$ and $H^{-1} \subseteq H$\\
		\item $HH^{-1} \subseteq H$
	\end{enumerate}
\end{proposition}

\vspace{2mm}

\begin{definition}
	Let $G$ be a group and $X \subseteq G$. Define

	\begin{equation}
		\left\langle X \right\rangle := \bigcap_{X \subseteq H \leq G} H
	\end{equation}
\end{definition}

\vspace{2mm}

\begin{proposition}
	Let $X$ be a subset of a group $G$. Then 

	\begin{equation}
		\left\langle X \right\rangle = \cbr[0]{ x_1 \cdots x_n : \forall i \in I \> x_i \in X \cup X^{-1}, n \in \mathbb{N} }
	\end{equation}
\end{proposition}

\subsubsection{Normal Subgroups}


%Homomorphisms
\subsection{Homomorphisms}

\begin{proposition}
	If $\varphi: A \to B$ is a group homomorphism, then $\varphi\del{1} = 1$, $\varphi\del[0]{x^{-1}} = \varphi\del{x}^{-1}$ and $\varphi\del[0]{x^n} = \varphi\del{x}^n$ for all $x \in A$ and $n \in \mathbb{Z}$.
\end{proposition}

\vspace{2mm}

\begin{proposition}
	A group homomorphism $\varphi: A \to B$ is injective if and only if $\ker\del{\varphi} = \cbr{1}$.
\end{proposition}

\vspace{2mm}

\begin{lemma}
	Let $\varphi: G \to H$ be an injective group homomorphism and let $a \in G$ be of finite order. Then $\abs[0]{\langle\varphi(a)\rangle} = \abs[0]{\langle a \rangle}$.	
\end{lemma}

\vspace{2mm}

\begin{proposition}
	If $G = \langle X \rangle$ and $\varphi, \psi: G \to G'$ are group homomorphisms with $\varphi\del{x} = \psi\del{x}$ for every $x \in X$ then $\varphi = \psi$. 
\end{proposition}

\vspace{2mm}



%Cyclic Groups
\subsection{Cyclic Groups}

\begin{definition}
	A group or subgroup is \emph{cyclic} when it is generated by a single element.
\end{definition}

\vspace{2mm}

\begin{lemma}
	Every group of prime order is cyclic.
\end{lemma}

\vspace{2mm}

\begin{proposition}
	Every subgroup of $(\mathbb{Z},+)$ is cyclic, generated by a unique nonnegative integer.
\end{proposition}

\begin{proof}
	Let $H \leq \mathbb{Z}$. If $H = \cbr[0]{0}$, then $H = \innerprod{0}$. So assume $H \neq \cbr[0]{0}$. Thus $H$ contains a nonzero integer. Since $H \leq \mathbb{Z}$ we have that $H$ contains a positive integer. Let us denote the smallest positive integer in $H$ by $n$. Every integer multiple of $n$ belongs to $H$. Conversly, if $m \in H$ we have by integer division $m = nq + r$ for some $q,r \in \mathbb{Z}$ with $0 \leq r < n$. So $m - nq = r \in H$ contradicting the minimality of $n$ being the smallest positive integer. Hence $m = nq$ and so $m \in \innerprod{n}$.
\end{proof}

\vspace{2mm}

\begin{proposition}
	Let $G$ be a group and let $a \in G$. If $a^m \neq 1$ for all $m \neq 0$, then $\langle a \rangle \cong \mathbb{Z}$; in particular $\langle a \rangle$ is infinite. Otherwise, there is a smallest positive integer $n$ such that $a^n = 1$, and then $a^m = 1$ if and only if $n$ divides $m$, and $\langle a \rangle \cong \mathbb{Z}/n\mathbb{Z}$; in particular, $\langle a \rangle$ is finite of order $n$.
\end{proposition}

\vspace{2mm}

\begin{corollary}
	Any two cyclic groups of order $n$ are isomorphic.
\end{corollary}

\vspace{2mm}

\begin{corollary}
	Every subgroup of a cyclic group is cyclic. Furthermore, either $H = \cbr[0]{1}$ or $H = \innerprod{x^n}$, where $n$ is the least positive integer with $x^n \in H$.
\end{corollary}

\begin{proof}
	Let $H \leq G := \innerprod{x}$, $H \neq \cbr[0]{1}$. Then $H' := \cbr[0]{k \in \mathbb{Z} : x^k \in H} \leq (\mathbb{Z},+)$ since 

	\begin{itemize}
		\item $0 \in H'$ by $1 = x^0 \in H \leq G$;
		\item If $k,k' \in H'$ then $k + k' \in H'$ by $x^{k + k'} = x^kx^{k'} \in H$;
		\item If $k \in H'$ then also $-k \in H'$ by $x^{-k} = \del[0]{x^{k}}^{-1} \in H$. 
	\end{itemize}

	Therefore $H' = \innerprod{n}$ for the least positive integer in $H'$. Now consider $\innerprod{x^n}$. We have that $\innerprod{x^n} \subseteq H$ by the previous observation and if $x^k \in H$ for some $k \in \mathbb{Z}$ we have $k \in H' = \innerprod{n}$ and so $k = mn$ for some $m \in \mathbb{Z}$ which yields $x^k = x^{mn} = \del[0]{x^n}^m \in \innerprod{x^n}$.    
\end{proof}

\vspace{2mm}

\begin{proposition}
	A cyclic group $G := \innerprod{x}$ of finite order $n$ has a unique subgroup of order $d$, namely $\langle x^{n/d} \rangle = \cbr[0]{g \in G : g^d = 1}$, for every divisor $d$ of $n$.	
\end{proposition}

\begin{proof}
	We prove the equality $\langle x^{n/d} \rangle = \cbr[0]{g \in G : g^d = 1}$ only. An element of $\langle x^{n/d} \rangle$ has the form $x^{kn/d}$ for some integer $k$. Therefore 

	\begin{equation*}
		\del[0]{x^{kn/d}}^d = (x^n)^k = 1^k = 1
	\end{equation*}

	Conversly if $g \in G$ with $g^d = 1$ we have that $g = x^k$ for some integer $k$ since $G$ is cyclic. Hence $1 = g^d = x^{kd}$ and so $n \mid kd$. So 

	\begin{equation*}
		g = x^k = (x^{kd/n})^{n/d} = (x^{n/d})^{kd/n} \in \langle x^{n/d}\rangle
	\end{equation*}
\end{proof}

\vspace{2mm}

\begin{lemma}
	For all $n \in \mathbb{N}$

	\begin{equation*}
		\mathbb{Z}^\times_n = \cbr[0]{\overline{k} :\gcd(n,k)=1} \qquad \text{and} \qquad \abs[0]{\mathbb{Z}^\times_n} = \varphi(n)
	\end{equation*}
\end{lemma}

%Symmetric Groups
\subsection{Symmetric Groups}
\subsubsection{Cycles}
\begin{lemma}
	Let 
\end{lemma}

%Automorphisms
\subsection{Automorphisms}

\begin{definition}
	Let $G$ be a group. For $a \in G$ define 

	\begin{equation}
		\iota_a: \begin{cases}
			G \to G\\
			x \mapsto axa^{-1}
		\end{cases}
	\end{equation}

	Further

	\begin{equation}
		\Inn(G) := \cbr[0]{\iota_a : a \in G}
	\end{equation}
\end{definition}

\vspace{2mm}

\begin{lemma}
	We have $\iota_a \in \Aut(G)$ for any $a \in G$. Furthermore $\Inn(G) \unlhd \Aut(G)$.
\end{lemma}

\begin{proof}
	$\iota_a \in \Aut(G)$ and $\Inn(G) \leq \Aut(G)$ is obvious. Let $\iota_a \in \Inn(G)$ and $\varphi \in \Aut(G)$. Then for $x \in G$ 

	\begin{equation*}
		\varphi\iota_a\varphi^{-1}(x) = \varphi(a \varphi^{-1}(x)a^{-1}) = \varphi(a)x\varphi(a)^{-1}
	\end{equation*}

	\noindent so $\varphi \iota_{a} \varphi^{-1} = \iota_{\varphi(a)} \in \Inn(G)$.
\end{proof}

\vspace{2mm}

\begin{lemma}
	The mapping 

	\begin{equation}
		\iota: \begin{cases}
			G \to \Aut(G)\\
			a \mapsto \iota_a
		\end{cases}
	\end{equation}

	\noindent is a homomorphisms. Furthermore $\ker(\iota) = Z(G)$.
\end{lemma}

\begin{proof}
	That $\iota \in \Hom(G,\Aut(G))$ is obvious. For $a \in \ker(\iota)$ we must have that $\iota_a = \id$, so $axa^{-1} = x$ for any $x \in G$. Hence $a \in Z(G)$. The converse is trivial.
\end{proof}

Thus by the first isomorphism theorem we have 

\begin{equation*}
	G/Z(G) = \Inn(G)
\end{equation*}

\begin{proposition}
	Let $G$ be a group and assume $G/Z(G)$ is cyclic. Then $G$ is abelian.
\end{proposition}

\begin{proof}
	
\end{proof}

%Direct Products
\subsection{Direct Products}
\begin{proposition}
	A group $G$ is isomorphic to the direct product $G_1 \times G_2$ of two groups $G_1$, $G_2$ if and only if it contains normal subgroups $A \cong G_1$ and $B \cong G_2$ such that $A \cap B = \cbr[0]{1}$ and $AB = G$.
\end{proposition}


%Latin squares

\begin{lemma}
	Let $G$ be a group. For every $a \in G$ the mappings

	\begin{equation}
		\vartheta_a: \begin{cases}
			G \to G\\
			x \mapsto ax
		\end{cases} \qquad \vartheta'_a: \begin{cases}
			G \to G\\
			x \mapsto xa
		\end{cases}
	\end{equation}

	\noindent are bijections.
\end{lemma}

\vspace{2mm}

\begin{theorem}
	
\end{theorem}

\vspace{2mm}

\begin{lemma}
	Let $G$ be a group. If $x^2 = 1$ for every $x \in G$ then $G$ is abelian.
\end{lemma}

\vspace{2mm}


\vspace{2mm}

\begin{proposition}
	In a finite group, the inverse of an element is a positive power of that element.
\end{proposition}



\vspace{2mm}


\vspace{2mm}

\begin{proposition}
	If $G = \langle X \rangle$ and the elements of $X$ are pairwise interchangeable then $G$ is abelian. Hence every cyclic group is abelian.
\end{proposition}

\vspace{2mm}

\begin{definition}
	Let $G$ be a group. The \emph{order of an element} $x \in G$ is defined by $\abs[0]{\langle x \rangle}$.
\end{definition}

\vspace{2mm}

\begin{definition}
	Relative to $H \leq G$ the \emph{left coset} of an element $x \in G$ is the subset $xH$ of $G$; the \emph{right coset} of an element $x \in G$ is the subset $Hx$ of G.
\end{definition}

\vspace{2mm}

\begin{proposition}
	The left cosets of $H \leq G$ constitute a partition of $G$ and so do the right cosets.	
\end{proposition}

\vspace{2mm}

\begin{proposition}
	The number of left cosets of a subgroup is equal to the number of right cosets.
\end{proposition}

\vspace{2mm}

\begin{definition}
	The \emph{index} $\intcc{G:H}$ of $H \leq G$ is the cardinal number of its left or right cosets.
\end{definition}

\vspace{2mm}

\begin{proposition}\emph{(Lagrange's Theorem)}
	If $H \leq G$, then $\abs{G} = \intcc{G : H}\abs{H}$. Hence if $\abs{G} < \infty$, the order and the index of a subgroup divide the order of $G$.
\end{proposition}

\vspace{2mm}

\begin{theorem}
	In a finite group $G$ we have $g^{\abs[0]{G}} = 1$ for any $g \in G$.
\end{theorem}

\vspace{2mm}

\begin{definition}
	Let $N \unlhd G$. The group of all cosets of $N$ is the \emph{quotient group} $G/N$ of $G$ by $N$. The homomorphism $x \mapsto xN = Nx$ is the \emph{canonical projection} of $G$ onto $G/N$.
\end{definition}

\vspace{2mm}

\begin{proposition}
	Let $N \unlhd G$. Every subgroup of $G/N$ is the quotient $H/N$ of a unique subgroup $H$ of $G$ that contains $N$.
\end{proposition}

\vspace{2mm}

\newpage
\section{Rings}

\begin{definition}
	An algebraic structure $(R,+,\cdot)$ with binary operations $+,\cdot: R \times R \to R$ is called a \hl{ring} if $(R,+)$ is an ableian group, $(R,\cdot)$ is a semigroup and for all $x,y,z \in R$ it holds that

	\begin{equation*}
		x(y + z) = xy + xz \qquad \text{and} \qquad (x + y)z = xz + yz.
	\end{equation*}
\end{definition}

\vspace{1mm}

\begin{definition}
	Let $R$ be a ring. A subset $S \subseteq R$ is called \hl{subring} if $(S,+) \leq (R,+)$ and $xy \in S$ for every $x,y \in S$. If $R$ is a ring with unity, then also $1 \in S$.
\end{definition}

\vspace{1mm}

\begin{definition}
	A commutative ring $R$ with unity is called an \hl{integral domain} if it has one of the following equivalent properties:

	\begin{enumerate}[label = \textup{(}\roman*\textup{)}]
		\item \emph{(Cancellation)} $zx = zy$ implies $x = y$ for any $x,y,z \in R$ with $z \neq 0$.
		\item \emph{(No divisors of zero)} $xy = 0$ implies either $x = 0$ or $y = 0$ for any $x,y \in R$.
	\end{enumerate}
\end{definition}

\vspace{1mm}

\begin{definition}
	A ring $R$ with unity is called a \hl{skew field} if $R^\times = R\setminus\cbr[0]{0}$.
\end{definition}

\vspace{1mm}

\begin{definition}
	A commutative skew field is called a \hl{field}.
\end{definition}

\vspace{1mm}

\begin{definition}
	A ring $R \neq \cbr[0]{0}$ is called \hl{simple} if $(0)$ and $R$ are the only ideals.
\end{definition}

\vspace{1mm}

\begin{definition}
	Let $R$ be a commutative ring. An ideal $P \neq R$ is called a \hl{prime ideal} if $ab \in P$ implies either $a \in P$ or $b \in P$ for $a,b \in R$.
\end{definition}

\vspace{1mm}

\begin{lemma}
	An ideal $P \neq R$ of a commutative ring $R$ with unity is a prime ideal if and only if $R/P$ is an integral domain.
\end{lemma}

\vspace{1mm}

\begin{definition}
	An ideal $M \neq R$ is called \hl{maximal} if there exists no ideal $I$ such that $M \subsetneq I \subsetneq R$.
\end{definition}

\vspace{1mm}

\begin{lemma}
	Let $R$ be a commutative ring with unity. An ideal $M \neq R$ is maximal if and only if $R/M$ is a field.
\end{lemma}

\vspace{1mm}

\begin{definition}
	An integral domain $R$ is called a \hl{factorial domain} or \hl{unique factorisation domain} when the following properties hold: 

	\begin{enumerate}[label = \textup{(}\roman*\textup{)}]
		\item Every element $x \notin R^\times \cup \cbr[0]{0}$ can be written as product of irreducible factors.

		\item If $p_1 \cdots p_n = q_1 \cdots q_m$ for irreducible $p_1,\dots,p_n,q_1,\dots,q_m \in R$, then $n = m$ and there exists $\sigma \in S_n$ such that $p_i \sim q_{\sigma(i)}$ for $i = 1,\dots,n$.
	\end{enumerate}
\end{definition}

\vspace{1mm}

\begin{definition}
	An integral domain is called a \hl{principal ideal domain} if every ideal of $R$ is principal.
\end{definition}

\vspace{1mm}

\begin{theorem}
	Every principal ideal domain is a factorial domain.
\end{theorem}

\vspace{1mm}

\begin{definition}
	An integral domain $R$ is called an \hl{euclidean domain} if there is a mapping $\varphi: R \setminus \cbr[0]{0} \to \mathbb{N}_0$ with the following property: To any $a,b \in R$ with $b \neq 0$ there exist $q,r \in R$ such that $a = qb + r$ and either $r = 0$ or $\varphi(r) < \varphi(b)$.
\end{definition}

\vspace{1mm}

\begin{theorem}
	Every euclidean domain is a principal ideal domain.
\end{theorem}

\vspace{1mm}

\begin{definition}
	A ring $R$ is called \hl{noetherian} if it has one of the following equivalent properties:

	\begin{enumerate}[label = \textup{(}\roman*\textup{)}]
		\item Every ascending chain $A_1 \subseteq A_2 \subseteq \dots$ of ideals $A_i$ of $R$ is stationary, i.e. there exists some $k \in \mathbb{N}$ such that $A_i = A_k$ for every $i \geq k$.
		\item Every nonempty collection of ideals of R contains a maximal element.
		\item Every ideal of $R$ is finitely generated.
	\end{enumerate}
\end{definition}

\vspace{1mm}

\begin{theorem}\emph{(Hilbert)}
	If $R$ is a commutative noetherian ring with unity then $R[X]$ is noetherian.
\end{theorem}

\newpage
%Usefull Stuff
\section{Usefull Stuff}
\begin{itemize}
	\item Let $G$ be a group and $H,K \leq G$. Then 

		\begin{equation}
			\sbr[0]{G:(H \cap K)} \leq \sbr[0]{G:H} \sbr[0]{G:K}.
		\end{equation}

	\item Consider the system of congruence equations

		\begin{equation}
			X \equiv a_1 \bmod r_1 , \dots , X \equiv a_n \bmod r_n
			\label{eq:congruences}
		\end{equation}

		\noindent where $r_1,\dots,r_n \in \mathbb{Z}$ are pairwise coprime and $a_1,\dots,a_n \in \mathbb{Z}$. Now set 

		\begin{equation}
			r := r_1 \cdots r_n \qquad \text{and} \qquad s_i := \frac{r}{r_i} 
		\end{equation}

		\noindent for each $i = 1,\dots,n$ and determine $k_i \in \mathbb{Z}$ such that 

		\begin{equation}
			k_is_i \equiv 1 \bmod r_i
		\end{equation}

		\noindent for each $i = 1,\dots,n$. This can be done using the extended euclidean algorithm, i.e. since $s_i$ and $r_i$ are coprime, we find $t_i \in \mathbb{Z}$ such that
		
		\begin{equation}
			k_is_i + t_ir_i = 1.
		\end{equation}
		
		Then 

		\begin{equation}
			k := k_1s_1a_1 + \dots + k_ns_na_n
		\end{equation}

		\noindent is a solution of (\ref{eq:congruences}) and the set of solutions of (\ref{eq:congruences}) is $k + r\mathbb{Z}$.

	\item Let $R$ be an integral domain with $p := \Char R \neq 0$. Then 

		\begin{equation}
			(a + b)^p = a^p + b^p
		\end{equation}

		\noindent holds for any $a,b \in R$ and for any $k \in \mathbb{N}$ by induction

		\begin{equation}
			(a + b)^{p^k} = a^{p^k} + b^{p^k}.
		\end{equation}

	\item A homomorphism of rings $\varphi: R \to S$ is called an \hl{embedding} if $\varphi$ is injective.

	\item $\abs[0]{\Aut \mathbb{Q}} = \abs[0]{\Aut \mathbb{R}} = 1$. First part is easy, second part follows from density of $\mathbb{R}$. Show that any homomorphism is order preserving.

	\item 
	\item $R$ PID, then for $0 \neq p \in R$ it holds that $(p)$ is a prime ideal if and only $(p)$ is maximal.

	\item 
\end{itemize}


\end{document}

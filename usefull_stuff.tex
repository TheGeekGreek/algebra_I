\section{Usefull Stuff}
\begin{itemize}
	\item Let $G$ be a group and $H,K \leq G$. Then 

		\begin{equation}
			\sbr[0]{G:(H \cap K)} \leq \sbr[0]{G:H} \sbr[0]{G:K}.
		\end{equation}

	\item Consider the system of congruence equations

		\begin{equation}
			X \equiv a_1 \bmod r_1 , \dots , X \equiv a_n \bmod r_n
			\label{eq:congruences}
		\end{equation}

		\noindent where $r_1,\dots,r_n \in \mathbb{Z}$ are pairwise coprime and $a_1,\dots,a_n \in \mathbb{Z}$. Now set 

		\begin{equation}
			r := r_1 \cdots r_n \qquad \text{and} \qquad s_i := \frac{r}{r_i} 
		\end{equation}

		\noindent for each $i = 1,\dots,n$ and determine $k_i \in \mathbb{Z}$ such that 

		\begin{equation}
			k_is_i \equiv 1 \bmod r_i
		\end{equation}

		\noindent for each $i = 1,\dots,n$. This can be done using the extended euclidean algorithm, i.e. since $s_i$ and $r_i$ are coprime, we find $t_i \in \mathbb{Z}$ such that
		
		\begin{equation}
			k_is_i + t_ir_i = 1.
		\end{equation}
		
		Then 

		\begin{equation}
			k := k_1s_1a_1 + \dots + k_ns_na_n
		\end{equation}

		\noindent is a solution of (\ref{eq:congruences}) and the set of solutions of (\ref{eq:congruences}) is $k + r\mathbb{Z}$.

	\item Let $R$ be an integral domain with $p := \Char R \neq 0$. Then 

		\begin{equation}
			(a + b)^p = a^p + b^p
		\end{equation}

		\noindent holds for any $a,b \in R$ and for any $k \in \mathbb{N}$ by induction

		\begin{equation}
			(a + b)^{p^k} = a^{p^k} + b^{p^k}.
		\end{equation}

	\item A homomorphism of rings $\varphi: R \to S$ is called an \hl{embedding} if $\varphi$ is injective.

	\item $\abs[0]{\Aut \mathbb{Q}} = \abs[0]{\Aut \mathbb{R}} = 1$. First part is easy, second part follows from density of $\mathbb{R}$. Show that any homomorphism is order preserving.

	\item 
	\item $R$ PID, then for $0 \neq p \in R$ it holds that $(p)$ is a prime ideal if and only $(p)$ is maximal.

	\item Let $d \in \mathbb{Z} \setminus \cbr[0]{1}$ be square-free. Then $\mathbb{Z}[\sqrt{d}] \subseteq \mathbb{C}$ is a commutative ring with identity. The mapping 

		\begin{equation}
			\overline{\cdot}:\begin{cases}
				\mathbb{Z}[\sqrt{d}] \to \mathbb{Z}[\sqrt{d}]\\
				x + y\sqrt{d} \mapsto x - y\sqrt{d}
			\end{cases}
		\end{equation}

		\noindent is an automorphism. Further, the mapping $N(x) := x\overline{x}$ is multiplicative.

\end{itemize}
